\documentclass{article} 
\usepackage[T1]{fontenc}
\usepackage[utf8]{inputenc}

\title{IPFS-Node} 
\author{Dean F. Valentine Jr.} 
\date{\today}

\begin{document}

\maketitle

\section{Introduction}

I have looked for a bit, and not found a FLOSS implemmentation of an extensible,
lightweight, remote HTTP proxy for IPFS instances. Well, here it is (in
experimental stages).  IPFS-Node lets you manage your remote IPFS instances by
sending authenticated requests to this server, which forwards them to the IPFS
HTTP API, IPFS Remote requires just a single RSA/ECDSA key in the x509 format,
and uses that to verify incoming JSON Web Tokens passed along with the requests,
which it transparently forwards to the IPFS instance configured.

\section{Installation}

IPFS-Node is not currently in the repositories of any mainstream linux
distributions, nor does it provide an installation binary for windows/macos.
Currently the only way to install IPFS-Node is to compile and run/schedule it
yourself, placing relevant configuration files in the local directory. This is
going to be remedied soon.

\section{Configuration}

IPFS-Node requires a configuration file located at config.json in the directory
where it is run. A sample configuration file is located in the github
repository and should be self explanatory.

\section{Security}

The IPFS-Node proxy runs only on HTTPS and thus requires a TLS keypair to
startup. IPFS-Node also requires one or more public keys for use in verifying
the signature of IPFS tokens. IPFS-Node supports all of the keys that the
golang standard library can parse from the X509 standard, namely ECDSA, RSA,
and DSA.


\end{document}

